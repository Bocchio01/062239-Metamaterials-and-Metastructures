\section{Conclusions}

\begin{frame}{Summary of results}

    In this work we have shown that in a 1D beam structure, \textbf{nonreciprocal behavior can be achieved and controlled} by modulating its properties in time and space via shunted piezoelectric actuation.

    \vspace{9pt}

    \textbf{The result is a diode-like behavior in the wave transmission}, where the wave is allowed to propagate in one direction while being attenuated in the opposite direction if a directional band-gap is present.

\end{frame}



\begin{frame}{Applications \& Future work}

    The capability to control wave propagation in a nonreciprocal manner opens up a wide range of potential applications in the field of phononic devices and advanced signal processing, such as:

    \begin{itemize}
        \item Acoustic waveguides for unidirectional energy transfer;
        \item Enhanced vibration control systems in engineering structures;
        \item Phononic devices for advanced signal processing.
    \end{itemize}

    \vspace{9pt}

    Future work may focus on refining the modulation strategies to achieve even greater control over wave propagation and exploring the scalability of this approach to other multidimensional structures such as plates and shells.

\end{frame}
